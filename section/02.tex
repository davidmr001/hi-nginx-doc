\section{起步}
\subsection{hi-project}
\texttt{hi-project}是一个辅助脚本,安装在\path	{/home/centos7/nginx/hi}目录中。它的用途是为用户创建“起步”代码模板。运行它可以使用三个选项,依次是:
\begin{itemize}
\item 工程名,可选,默认\texttt{demo}
\item 工程类型,可选,支持\texttt{cpp},\texttt{python}和\texttt{lua},默认\texttt{cpp}
\item \texttt{hi-nginx}安装路径,可选,默认\path{/home/centos7/nginx}
\end{itemize}
用户可以通过\texttt{-h}或者\texttt{--help}参看使用说明。
\subsection{hello world}
\texttt{hello world}工程包含了\texttt{hi-nginx} \texttt{web application}开发的最基本要素。
\subsubsection{cpp}
首先使用\texttt{hi-project}脚本创建一个\texttt{cpp}工程,名为hello:
\begin{lstlisting}
/home/centos7/nginx/hi/hi-project hello cpp /home/centos7/nginx
\end{lstlisting}
后面两个参数是可省的。
这时,\texttt{hi-project}会创建一个名为hello的目录,并在该目录中创建两个文件,一个是\path{Makefile},一个是\path{hello.cpp}。前者帮助用户在执行\texttt{make \&\& make install }时把\texttt{web application}编译、安装至正确位置;后者则帮助用户正确创建合乎\texttt{hi-nginx}要求的\texttt{class}:
\begin{lstlisting}
#include "servlet.hpp"


namespace hi {

    class hello : public servlet {
    public:

        void handler(request& req, response& res) {
            res.headers.find("Content-Type")->second = "text/plain;charset=UTF-8";
            res.content = "hello,world";
            res.status = 200;
        }

    };
}

extern "C" hi::servlet* create() {
    return new hi::hello();
}

extern "C" void destroy(hi::servlet* p) {
    delete p;
}
\end{lstlisting}
以上代码一目了然,无需过多解释,任何熟知\texttt{c++}的程序员都能看懂。没错,\texttt{hi-nginx}并不要求\texttt{cpp}程序员“精通”自己的工具,只需熟知即可。当然,熟知\texttt{http}协议是必要的,否则很难正确地使用\texttt{request}类和\texttt{response}类。

\texttt{request}类包含了客户端访问\texttt{hi-nginx}	时所携带的信息。

\subsubsection{python}
\subsubsection{lua}





