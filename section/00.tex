\section{导言}
\texttt{hi-nginx}既是\texttt{web server},也是\texttt{application server}。这是它区别于\texttt{nginx}的最主要的特点。作为前者,它跟\texttt{nginx}一样,可作静态资源服务器,可作反向代理服务器,还可作负载均衡服务器,一切都一如\texttt{nginx}。作为后者,它让\texttt{c++}程序员,\texttt{python} 程序员,以及\texttt{lua}程序员写的\texttt{web application}完全运行在\texttt{nginx}服务器内部,从而可以轻松提供高性能的、支持大并发的\texttt{web application}。

为什么\texttt{hi-nginx}要同时支持三种编程语言?其原因有三。第一,我最常用的编程语言是\texttt{c++},它必须被支持;而且它非常快,非常适合处理“热点”业务。第二,\texttt{python}库资源极为丰富,非常适合处理常规业务,几乎没有它未曾涉猎的开发领域,因而它能够极大地加快开发速度。第三,\texttt{lua}比\texttt{python}快,但是库资源不及后者,支持它是为了方便处理那些介于"热点"业务和“常规”业务之间的业务。

因此,使用\texttt{hi-nginx},让其发挥出最大潜能,需要开发者同时熟知\texttt{c++}、\texttt{python}、\texttt{lua}三种编程语言。这并不是非常高的要求。实际上,这三种语言都非常易学易用,尽管并不是所有人都认同这一点。当然,用户只熟知其中的某一种编程语言也无妨——即便是对\texttt{python}程序员而言,\texttt{hi-nginx}	也能提供非常高效的并发处理能力。

目前,把\texttt{c}或者\texttt{c++}运用于\texttt{web}应用开发的最主要的方式是\texttt{script}加\texttt{c}或者\texttt{c++} \texttt{extension}。这样做的目的其实主要地还是为了解决\texttt{script}可能无法胜任“热点”业务的问题。首先是脚本,然后是\texttt{c}或者\texttt{c++} 扩展,最后再回到脚本。这条性能优化路线在\texttt{web}应用开发中极为常见。\texttt{hi-nginx}不仅支持这条路线——用户照样可以为\texttt{python}和\texttt{lua}开发\texttt{c}或\texttt{c++}扩展——而且还支持另一条路线,即直接用\texttt{c++}写\texttt{web application}。这条路线省去了“脱裤子”的麻烦;对于能写\texttt{c}或\texttt{c++}扩展的程序员而言,这是极为便利的。当然,如果用户仅仅能写脚本,\texttt{hi-nginx}也保证提供比已有的反向代理方案更强大的并发能力。

\texttt{hi-nginx}致力于增强用户的工作,而不是改变用户的工作。当用户不满意它时,用户可以安全地“回滚”至之前的工作状态,而不会产生任何损失。