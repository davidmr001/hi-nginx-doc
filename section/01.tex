\section{快速部署}
\texttt{hi-nginx}目前仅仅支持\texttt{Linux}系统。

\subsection{下载源码包}
\texttt{hi-nginx}是一个开源在\url{https://github.com/webcpp/hi-nginx}上的一个开源项目。用户可以直接从该地址下载\texttt{hi-nginx}的源码包。建议下载最新的正式发布版,前往\url{https://github.com/webcpp/hi-nginx/releases}查看并点击下载最新版本即可。

不建议直接使用\texttt{git clone}命令下载未正式分布的源码包。

\subsection{编译及安装}
要编译\texttt{hi-nginx},需要安装一些依赖软件。包括:
\begin{itemize}
\item gcc
\item gcc-c++
\item make
\item pcre-devel
\item zlib-devel
\item openssl-devel
\item hiredis-devel
\item python-devel
\item boost-devel
\item luajit-devel
\end{itemize}
假设用户使用的\texttt{Linux}是\texttt{CentOS},那么很简单,执行:
\begin{lstlisting}
sudo yum install gcc gcc-c++ make pcre-devel zlib-devel openssl-devel hiredis-devel python-devel boost-devel luajit-devel
\end{lstlisting}
即可。

安装以上依赖软件之后,解压缩\texttt{hi-nginx}源码包。进入源码目录后,会看到一个演示性的安装脚本\texttt{install-demo.sh}。如果用户仅仅是尝试\texttt{hi-nginx},可以直接运行该该脚本。它的内容很简单:
\begin{lstlisting}
#!/bin/bash
./configure     --with-http_ssl_module \
                --with-http_v2_module \
                --prefix=/home/centos7/nginx \
                --add-module=ngx_http_hi_module
\end{lstlisting}
这个脚本告诉用户,编译\texttt{hi-nginx}与编译\texttt{nginx}没有什么不同,只需使用配置选项\texttt{--add-module}指定\texttt{ngx_http_hi_module},其他则一如后者。

执行完以上步骤之后,就只剩下\texttt{make \&\& make install}了。安装目录是\texttt{--prefix}选项指定的\path{/home/centos7/nginx}。


\subsection{基本测试}
现在,\texttt{hi-nginx}已经安装到了\path{/home/centos7/nginx}目录中。用户可以从该目录启动\texttt{hi-nginx}。启动方法与\texttt{nginx}无异。

但是,既然已经安装好了\texttt{hi-nginx},在正式使用它之前,就应该测试一下它\texttt{application server}功能,确保安装的成功。为此,用户需要做两件事情。第一件事情是安装\texttt{redis}服务器,因为\texttt{hi-nginx}的会话功能需要它。安装的方法也很简单:\texttt{sudo yum install redis}。第二件事情是下载\url{https://github.com/webcpp/hi_demo}上的演示代码。\texttt{git clone https://github.com/webcpp/hi_demo.git}即可。演示代码假定\texttt{hi-nginx}的安装目录为\path{/home/centos7/nginx},执行\texttt{make \&\& make install}即可。如果安装目录不同于此,需修改\texttt{hi_demo}目录下\texttt{Makefile}中\texttt{NGINX_INSTALL_DIR}变量的值。

\texttt{hi_demo}目录下有两个特殊文件,一个是\texttt{demo.html},一个是\texttt{demo.conf}。前者可带领用户测试\texttt{hi-nginx},后者则用户配置\texttt{hi-nginx}以正确加载\texttt{hi_demo}。
用户执行以下命令就可以正确安装这两个文件:
\begin{lstlisting}
install demo.html /home/centos7/nginx/html
install demo.conf /home/centos7/nginx/conf
\end{lstlisting}
完成以上步骤之后,就可以正式测试\texttt{hi-nginx}了。

进入\texttt{hi-nginx}安装目录,执行\texttt{sbin/nginx -c conf/demo.conf},然后访问\url{http://localhost:8765/demo.html},按链接指引点击即可。如无意外,\texttt{hi-nginx}会通过所有测试。如果遇到问题,用户可查看\path{/home/centos7/nginx/conf/demo.conf}文件:
\begin{lstlisting}
        hi_need_cache on;			#缓存开关
        hi_cache_size 10;			#缓存容器大小
        hi_cache_expires 300s;		#缓存过期时间
        hi_need_headers off;		#http header 开关
        hi_need_cookies off;		#http cookie 开关
        hi_need_session off;		#http session 开关
        hi_session_expires 300s;	#http session 过期时间
        hi_redis_host 127.0.0.1;	#redis 主机
        hi_redis_port 6379;		#redis 端口
        
        expires 10s;
        location = /hello {
            hi_need_cache off;
            hi hi/hello.so;
        }



        location ^~ /form {
            rewrite ^/form/(\d+)$ /form/?item=$1 break;
            hi_need_cache off;
            hi_need_headers on;
            hi_need_cookies on;
            hi hi/form.so;
        }

        location = /error {
            hi hi/error.so;
        }

        location = /redirect {
            hi hi/redirect.so;
        }

        location = /empty {
            hi hi/empty.so;
        }

        location = /math {
            hi_cache_expires 5s;
            hi hi/math.so;
        }
        
        location = /session {
            hi_need_cache off;
            hi_need_session on;
            hi_session_expires 30s;
            hi hi/session.so;
        }

	location = /pyecho {
            hi_need_cache off;
            hi_python_content "hi_res.status(200)\nhi_res.content('hello,world')";		#运行python内容块
	
	}

	location ~ \.py$ {
            hi_need_cache off;
            hi_need_headers on;
            hi_need_session on;
            hi_session_expires 30s;
            hi_python_script python;											#运行python脚本
	}

	location = /luaecho {
            hi_need_cache off;
            hi_lua_content "hi_res:status(200)\nhi_res:content('hello,world')";		#运行lua内容块
	
	}

	location ~ \.lua$ {
            hi_need_cache off;
            hi_need_headers on;
            hi_need_session on;
            hi_session_expires 30s;
            hi_lua_script lua;													#运行lua脚本
	}

        location / {
            root   html;
            index  index.html index.htm;
        }
\end{lstlisting}
对照以上配置检查哪里出现不意状态。也可查看\path{logs/error.log},看看有何种提示信息。








